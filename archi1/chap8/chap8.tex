% Tizedik előadás

\chapter{Memóriák}

A memóriák nagyságrendekkel gyorsabbak a merevlemezeknél, de jelentősen lassabbak a gyorsítótáraknál.

Csoportosításuk:
\begin{itemize}
    \item írható-olvasható (RAM)
    \begin{itemize}
        \item DRAM (operatív tár)
        \item SRAM (gyorsítótárak)
    \end{itemize}
    \item főképp csak olvasható (CMOS)
    \item csak olvasható (ROM)
\end{itemize}

\section{ROM}
Ma általában flash tárolóval oldják meg, tartalmuk maradandó.

\section{RAM}
Random Access Memory, tartalma nem maradandó.

\subsection{SRAM}
Statikus memória, a tápfeszültség megszűnéséig tárolja az adatot.
Félvezető, flip-flop memóriában tárolja az adatot, általában regisztereknek és cache-nek használják.
Jellemzői:
\begin{itemize}
    \item energiatakarékos
    \item gyors
    \item drága
    \item az adatokat nem kell frissíteni
\end{itemize}
A dinamikus RAM-nál nagyságrendekkel gyorsabb.

\subsection{DRAM}
Néhány pikofarados kondenzátorokból és 1 tranzisztorból épül fel egy cella.
Ez egy idő után kisül, így az adatot frissíteni kell.
Jellemzői:
\begin{itemize}
    \item olcsó
    \item alacsony fogyasztás
    \item kis helyigény
\end{itemize}

\subsubsection{Típusai}
\begin{itemize}
    \item klasszikus, aszinkron DRAM
    \item szinkron DRAM
    \begin{itemize}
        \item SDR - Single Data Rate: az órajelnek csak a felmenő élén történik átvitel
        \item DDR - Double Data Rate: felmenő és lemenő élre is történik adatátvitel
    \end{itemize}
\end{itemize}

\subsubsection{Felépítése}
A memória több logikai részre - bankokra - van felépítve, hogy egyszerre több memória műveletet is végre tudjon hajtani.
Olvasása futószalag elven történik, tehát a memória művelet parancs kiadása után az eredmény csak több óraciklus után jelenik meg.
Ebből fakad a késleltetés, azaz a latency, ami a memória egy fontos teljesítmény paramétere.

\subsection{Prefetch eljárás}
A prefetch eljárás fajtája megmutatja, hogy adott óraciklus alatt hány bit tölthető az IO pufferbe.
2n eljárásnál pl. ez a szám 2.

\subsubsection{Fejlődése}
\begin{itemize}
    \item DDR-400: 2n prefetch
    \item DDR2: alacsonyabb órafrekvencia, alacsonyabb fogyasztás, 4n prefetch
    \item DDR3: 8n prefetch
    \item DDR4: ugyanolyan órafrekvencia mellett nagyobb a késleltetés, mint DDR3-nál. Ugyanúgy 8n prefetch-t használ, viszont létrehoztak bankcsoportokat. Megjelenik a CRC és a chipenkénti extra paritás, így növekedett a stabilitás.
\end{itemize}

\subsubsection{Az olvasás lépései}
\begin{itemize}
    \item bank megnyitása
    \item oszlopblokkok olvasása a megnyitott sorból
    \item bank zárása
    \item legalább tRP várakozás, mielőtt a bankot újra megnyithatnánk
\end{itemize}
Tehát a legjobb az, ha a megnyitott bankok soraiból olvasunk.

\section{Fejlődési trendek}
\begin{itemize}
    \item nő az órafrekvencia és a sávszélesség
    \item nő a prefetch
    \item csökken a lapkaméret és a tápfeszültség
\end{itemize}

\section{DIMM-ek}
\begin{itemize}
    \item 64 bites szervezés: 168-284 pin
    \item registered DIMM: a DRAM chipek és a memóriasín közé helyeznek egy regisztert (ami minden műveletet pufferel), így kisebb elektromos terhelést jelent a vezérlő számára és több modul esetén nagyobb a stabilitás, ezért elsősorban szerverekhez használják.
    \item ECC chip: Error Control Coding, míg a paritás bit segítségével a memória felfedezheti az egyszeres hibát, viszont hibajavításra nem képes, az ECC chip segítségével javítani is tudja az adatot, valamint két bit hibát is képes felismerni.
    \item PLL chip: fázis zárt hurok, az órajel elcsúszás mentesítésre szolgál, ezt is elsősorban szerverekben használják
\end{itemize}

