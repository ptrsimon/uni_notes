% Tizenkettedik és tizenharmadik előadás

\chapter{Megszakítások}

A megszakítás célja a feldolgozás vagy végrehajtás közben váratlannak tekinthető események kezelése.
Nem csak a reagálás a cél ezekre az eseményekre, hanem a folyamatosan változó körülmények között az optimális működés biztosítása is.

\section{Általános folyamata}
Minden utasítás végrehajtás után van egy utasítás töréspont, amikor a processzor megvizsgálja, hogy érkezett-e megszakítás.
Ha a válasz nem, akkor a követkző utasítás következik.
Ha viszont igen, a processzor megvizsgálja, hogy a megszakítás elfogadható-e.
Amennyiben nem elfogadható, folytatja a következő utasítással, viszont ha igen, akkor megvizsgálja az adott megszakítást, menti az adott állapot kontextusát, kiszolgálja a megszakítást, visszatölti a mentett kontextust és jön a következő utasítás.

Fontos szempont, hogy megszakítás esetén az adott folyamat kontextusát tárolni kell.
Ezután be kell tölteni a megszakítási rutin állapotát, végre kell hajtani a megszakítást és végül vissza kell adni a vezérlést.

\section{Kialakulása}
Kezdetben az IO adatátvitel gyorsítására alkalmazták a megszakításokat.

\section{Megszakítási okok prioritási sorrendben}
\begin{enumerate}
    \item Géphibák: hibafigyelő áramkörök jelzései - általában nem letilthatók (pl. hőmérséklet szenzorok jelzései, ventillátor meghibásodása, paritás ellenőrzés, watchdog). Egy részük hibajelző kódok segítségével kommunikál.
    \item IO források: IO befejezés, IO kezdeményezés, minden perifériákkal kapcsolatos állapotjelzés
    \item Külső források: reset gomb, hálózati kommunikáció
    \item Programozási (szoftveres) források
    \begin{itemize}
        \item szándékos: pl. rendszerhívás
        \item hibakezelés: mindig valamilyen utasítás végrehajtása során alakul ki 
    \end{itemize}
\end{enumerate}

\section{Szoftveres megszakítások}
Hibákból származó programozási megszakítások lehetnek:
\begin{itemize}
    \item memóriavédelem megsértése (több, saját memóriaterülettel rendelkező program esetén)
    \item tényleges tárkapacitás túlcímzése
    \item címzési előírások megsértése pl. 32 bites címek esetén csak minden negyedik byte címezhető
    \item aritmetikai, logikai végrehajtás közben fellépő hibák (underflow, overflow, tartomány túllépése, nullával osztás...)
\end{itemize}

\section{Megszakítások csoportosítása}
Megkülönböztetünk szinkron és aszinkron megszakításokat.
A szinkron megszakítás egy adott program ugyanazon paramétereivel történő futtatása során mindig ugyanott jelentkezik (várható megszakítás).
Az aszinkron megszakítások ezzel szemben viszont véletlenszerűen lépnek fel (nem várható megszakítás, pl. hardverhiba).

Csoportosíthatjuk a megszakításokat még aszerint is, hogy az utasítások végrehajtása között (pl. utasítás végrehajtásának eredménye képpen) vagy közben lépnek fel.
Utasítások között felléphetnek pl. túlcsordulási hibák.
Ezek kezelése rögtön az utasítás után elkezdődhet.
A program folytatása a megszakítás kezelésének eredményétől függ.

Az utasítások közben felmerülő megszakítások nincsenek szinkronban a végrehajtási ciklussal.
Ilyenek pl. a hardvermegszakítások.
Ezek kezelése is az utasítás után kezdődhet meg.

A megszakításokat feloszthatjuk még aszerint is, hogy a programozó kérte, vagy nem kérte az adott megszakítást.

A megszakítás után a megszakított program folytatódhat, vagy be is fejeződhet.

További csoportosítás a maszkolható és nem maszkolható megszakítások.
A maszkolható megszakítások letilthatók.
Tipikus nem maszkolható megszakítások a hardverhibákat jelzők.

\section{Megszakítás kiszolgálása}
\begin{enumerate}
    \item Előkészítés: ha valamelyik egység megszakítás kérést bocsát ki, aktiválódik az INTR vezérlővonal (interupt request)
    \item Az első kérdés, hogy a megszakítás elfogadható-e. Régen a megszakítás vonalon csak egyszintű megszakítás lehetett, de ma már ez egy megszakítás áramkör, ami fogadja a bejövő megszakításokat és tudja kezelni a prioritásokat. Ez az áramkör van rákötve a CPU interrupt bementére, feladata eldönteni, hogy melyik megszakítás legyen kiszolgálva.
    \item A megszakítás érvényre juthat, ha:
    \begin{enumerate}
        \item az aktuális folyamat (program vagy másik megszakítás) megszakítható
        \item megfelelő a prioritás nagysága
        \item az adott megszakítás nincs maszkolva, azaz letiltva
    \end{enumerate}
    \item Ha a megszakítást elfogadta a megszakítás áramkör, az INTACK vezérlővonallal jelzi az elfogadást és aktiválja az INTR vonalat
\end{enumerate}

\subsection{A processzor feladata}
\begin{itemize}
    \item eltárolja az aktuális kontextust egy programtól független tárolóban
    \item betölti a PC-be a megszakítást kezelő utasítások kezdőpontját
\end{itemize}

\section{Megszakításrendszerek szintjei}
\begin{itemize}
    \item egyszintű
    \item többszintű
    \begin{itemize}
        \item egyvonalú
        \item többvonalú
    \end{itemize}
\end{itemize}

\subsection{Egyszintű megszakítási rendszer}
Egyszintű megszakításoknál addig nem kezdődhet másik megszakítás, amíg vissza nem térünk a normál állapotba.
Tehát ilyenkor a megszakítások nem megszakíthatók, a megszakítások egymás után hajtódnak végre.
Hátrány, hogy a magasabb prioritású megszakításoknál is előfordulhat, hogy várakozniuk kell.

\subsection{Egyvonalú, többszintű megszakítási rendszer}
Ennél a megszakítások is megszakíthatók, a megszakítások fontossági sorrendben futnak le.
Ha egy megszakítás futása során egy magasabb prioritású megszakítás érkezik, az alacsonyabb prioritású megszakítást megszakítja a rendszer és elkezdi végrehajtani a magasabb prioritásút.
A rendszer problémája, hogy a valóságban sok megszakítás van, ezért nem lehet minden megszakításhoz különböző prioritást rendelni.

\subsection{Többszintű, többvonalú megszakítási rendszer}
Fejlesztés az egyvonalú rendszerekhez képest, hogy megszakítási osztályokat hoznak létre és ezekhez az osztályokhoz rendelnek prioritási szinteket.
Az osztályokon belül a megszakítástípusok egyszintű prioritást használnak.

\section{Összefoglalás}
A megszakítások lényege, hogy csak akkor terhelje a processzort, amikor az szükséges, de akkor minnél hamarabb alkalmazkodni tudjon a rendszer a felmerült körülményhez.