% Tizedik előadás

\chapter{I/O rendszer}

A CPU-t és a perifériákat összekötő egységet hívjuk I/O egységnek.
Például egy billentyűzet csatlakozásánál az alaplapi USB vezérlő számít I/O egységnek.

\section{Fejlődésük}
\begin{enumerate}
    \item A CPU közvetlenül vezérli a perifériát (megszakítás nélküli programozott IO - wait for flag)
    \item Egy I/O modul kerül kialakításra
    \begin{itemize}
        \item megszakítás nélkül, wait for flag
        \item megszakításos vezérlés
    \end{itemize}
    \item DMA segítségével közvetlen hozzáférés a perifériához
    \item IO csatorna - lassabb perifériák számára. IO-ra specializált utasításkészlettel rendelkezik és a központi operatív tárat használja
    \item IO processzor - az IO csatorna továbbfejlesztése, saját működésre képes egység, saját memóriával
\end{enumerate}
A megszakítás nélküli vezérlés nagy hátránya, hogy az utasítás kiküldése után a CPU egy ciklusban várakozik a periféria válaszára, és amíg ez meg nem érkezett, blokkolta a további műveleteket.
Ezért jelentek meg a megszakításrendszerek, a processzor az utasítás kiadása után más feladatokkal foglalkozott, majd az IO egység megszakítással jelzi a művelet befejezését.
Következmény, hogy nő a teljesítmény.