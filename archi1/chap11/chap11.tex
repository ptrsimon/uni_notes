% Tizenharmadik előadás

\chapter{A tranzisztorok fejlődése}

\section{Bevezetés}
A processzorokat szilícium szeletek, "waferek" felhasználásával gyártják.
A waferek mérete folyamatosan növekszik a gyártástechnológia fejlődésével.

Egy integrált áramkör létrehozása 2-300 lépésből áll, kb 2-3 hónap.
A gyártáshoz speciális gyárra van szükség.

A tranzisztorokat a minimális csíkszélességgel jellemezhetjük.
A csíkszélesség mást jelenthet gyártónként, kisebb eltérések lehetnek.

A csíkszélesség lassan eléri a szilícium atomi méretét, ahol korlátokba ütközik a további csökkentés.

\section{Moore törvény}
Moore szerint a processzorokban a tranzisztorok száma két évente duplázódik, a csíkszélesség pedig csökken.
Ma már laposodik a görbe.

\section{Tranzisztorok típusai}
\begin{itemize}
    \item MOSFET: régebben használták őket, nagy frekvencián viszont nagyon melegedtek, ezért a teljesítmény nem volt tovább növelhető.
    \item feszített szilícium (2006-2009)
    \item HKMG (2012)
    \item FinFET (2014)
\end{itemize}
A fejlődés során a szükséges feszültség és a csíkszélesség is csökkent.

\section{MOSFET tranzisztorok}

\section{Feszített szilícium technológia}

\section{HKMG tranzisztorok}

\section{FinFET tranzisztorok}
A FinFET tranzisztorokat Tri-gate vagy 3D tranzisztoroknak is hívják.

Hasonló elven működig, mint a HKMG tranzisztorok, de a kialakítása változott.

A FinFET tranzisztorok teljesítménye kb. kétszerese a hagyományos tranzisztoroknak, a fogyasztás pedig kb. a 25-öd része.
Az 1 Wattra jutó teljesítmény a hagyományos NMOS tranzisztorokhoz képest 7-8-szoros.

\section{Slew rate}
A slew rate (meredekségi ráta) adja meg a tranzisztoroknál a jelváltozás meredekségét (mennyi idő alatt vált kikapcsoltból bekapcsolt állapotba a tranzisztor).