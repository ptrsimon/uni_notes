% Első előadás

\chapter{Alapfogalmak}

\section{Architektúra fogalma}
A számítógép architektúra fogalmat először Amdahl, az IBM mérnőke használta először a 360-as család bejelentésekor.
Definíciója szerint ez az a struktúra, amit a gépi kódú programozónak értenie kell, hogy helyes programot tudjon írni az adott gépre.
Tehát a regiszterek, memória, utasításkészlet, címzési módok és utasításkódok összessége, mind logikai, mind hardveres szinten.

\section{Számítási modellek}
A számítási modell a számításra vonatkozó alapelvek egy absztrakciója.
A számítási modelleket a következő absztrakciós jellemzőkkel írhatjuk le:
\begin{itemize}
    \item min hajtjuk végre a számítást (általában adatokon - adat alapú)
    \item hogyan képezzük le a számítási feladatot
    \item milyen módon vezéreljük a végrehajtási sorrendet
\end{itemize}

\subsection{A számítási modell, az architekrúra és a programnyelv kapcsolata}
Egy számítógép tervezését a számítási modellel kell kezdeni, ami meghatározza, hogy mit szeretnénk csinálni.
Ehhez szükség van egy specifikációs eszközre, amit a programnyelv képvisel (pl. Neumann modell megvalósítási eszköze a BASIC, Fortran).
Ezután jön az architektúra, ami a számítási modell implementációs eszköze, a "vas".
Ez hajtja végre az adott programnyelven definiált feladatokat.

\subsection{Számítási modellek csoportosítása}
\subsubsection{Számítási modelljük szerint}
\begin{itemize}
    \item szekvenciális
    \item párhuzamos
\end{itemize}

\subsubsection{Vezérlés meghajtása szerint}
\begin{itemize}
    \item vezérlés meghajtott
    \item adat meghajtott
    \item igény meghajtott
\end{itemize}

\subsubsection{Probléma leírása szerint}
\begin{itemize}
    \item procedurális
    \item deklaratív
\end{itemize}

Első sorban aszerint különböztetjük meg őket, hogy min hajtjuk végre a számítást.
Az adatalapú modellek:
\begin{itemize}
    \item Neumann modell
    \item adatfolyam modell
    \item applikatív modell (igénymeghajtott)
\end{itemize}
Az adatalapú modelleken kívül léteznek még objektum alapú, predikátum logika alapú, tudás alapú és hibrid modellek.
A mai processzorokban a Neumann és az adatfolyam modellek keverednek.

\subsection{Adatalapú modellek közös tulajdonságai}
\begin{itemize}
    \item az adatok általában típussal rendelkeznek (pl. 16 bit int) - vannak elemi és összetett adattípusok
    \item a típus meghatározza az adat értelmezési tartományát, értékkészletét és az elvégezhető műveleteket
\end{itemize}

\subsection{Neumann modell}
A Neumann-elvű számítógépek a számításokat adatokon hajtják végre, amiket egy változó értékű változókészlet képvisel.
A végrehajtási sorrend vezérlés meghajtott, tehát van egy statikus utasításszekvencia, amit egy speciális regiszter biztosít (program counter).
A program counter egy inkrementálódó változó, mindig a végrehajtandó utasításra mutat.
A végrehajtási sorrendtől vezérlési feladatokat ellátó utasításokkal lehet eltérni (pl. jump, if).

A Neumann elv követelményrendszere előírja változók létrehozását, adatmanipulációs és vezérlés átadási utasítások deklarálását.
Az ilyen nyelveket hívjuk imperatív (parancs) programnyelveknek (pl. C, Pascal, Assembly).

Ezeket a követelményeket az architektúra kielégíti, pl. lehetővé teszi, hogy a memóriában elhelyezkedő változók korlátlan számban módosíthatók legyenek a program futása során.
Ezen kívül biztosítja a megfelelő regisztereket az adatoknak és speciális regisztereket mint pl. program counter.

Az adatok és az utasítások a memóriában helyezkednek el.
A számítási feladat műveletek elemi műveletek sorozataként értelmezhető.
Egy számítási feladat leképezhető adat manipuláló utasítások sorozatával.
Az adat manipuláló utasítások az utasítások sorrendjében vannak végrehajtva, ezért ez egy vezérlés meghajtott modell.
A vezérlést a program counter biztosítja, a sorrendet a programozó határozza meg.
Az explicit vezérlés átadó utasításokkal lehet eltérni az implicit szekvenciától.

Következmények:
\begin{itemize}
    \item előzmény érzékenység: mivel az adatok változhatnak bármikor a végrehajtás során, a végrehajtás sorrendje nem mindegy
    \item alapvetően szekvenciális végrehajtást biztosít
    \item egyszerűen implementálható
    \item az adatmanipuláló utasítások nem szándékos állapotmódosulást okozhatnak (pl. overflow) - ezeket mellékhatásoknak hívjuk, kezelni kell őket
\end{itemize}

\subsection{Adatfolyam modell}
A számítást itt is adatokon hajtjuk végre, de:
\begin{itemize}
    \item az adatokat bemenő adathalmaz képviseli
    \item egyszeres értékadás lehetséges
    \item a megoldandó feladatot adatfolyam gráffal és input adatok halmazával képezzük le
    \item szakosodott végrehajtó egységeket használ
    \item a végrehajtást az adat vezérli - adatvezérelt, azaz az adat rendelkezésre állásakor azonnal működésbe lép a végrehajtó egység
\end{itemize}