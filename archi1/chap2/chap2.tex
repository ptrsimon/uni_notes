% Második és harmadik előadás

\chapter{Adattér}

\section{Az adattér fogalma}
Az adattér egy olyan tér, mely biztosítja az adatok tárolását oly módon, hogy azok a CPU által közvetlenül manipulálhatók legyenek.
Az adattér közvetlenül címezhető a processzor által.

\section{Típusai}
Az adattér két típusát különböztetjük meg:
\begin{itemize}
    \item memóriatér
    \item regisztertér
\end{itemize}

\section{Memóriatér}
A memóriateret leginkább a mérete jellemzi.
A címzéséhez címbuszra van szükség, ennek szélessége meghatározza a memóriatér maximális méretét.
A címtérnél megkülönböztetjük a modell címterét és a valós címteret, mivel az adott installáció nem feltétlen éri el az elméleti modell méretét.

\subsection{Virtuális memória}
A 60-as években a kis memóriák és az elméleti és valós memóriatér eltérő mérete miatt megjelent a virtuális memória.
Alap koncepciói:
\begin{itemize}
    \item két különböző memóriatér létezik: amit a programozó lát (virtuális memória) és a CPU által látott fizikai memória
    \item létezik olyan, a felhasználó számára transzparens folyamat, amely a program futása közben az éppen nem használt adatokat a valós memóriából a virtuális memóriába mozgatja, majd szükség esetén visszateszi
    \item létezik olyan (egyirányú) folyamat, amely a virtuális memória címeket dinamikusan, tehát futási időben valós címekké alakítja
\end{itemize}

\section{Regisztertér}
A regisztertér az adattér nagy teljesítményű, általában viszonylag kis része.
Általában nem része a címtérnek.

\subsection{Típusai}
\begin{itemize}
    \item egyszerű regiszterkészlet
    \item adattípusonként különböző regiszterek
    \item többszörös regiszterkészlet
\end{itemize}

\subsection{Egyszerű regiszterkészlet}
Az egyszerű regiszterkészlet típusai:
\begin{itemize}
    \item egyetlen regiszter (akkumulátor), hátránya, hogy nagyon lassú
    \item több, dedikált adatregiszter
    \item univerzális regiszterkészlet, jelentős teljesítmény növekedést eredményezett, a programozó szabadon felhasználhatja a regisztereket (pl. a gyakran használt változók folyamatosan a regiszterben maradhattak)
    \item stack regiszterek
\end{itemize}

\subsubsection{Stack regiszterek}
A stack regisztereknél egy stack pointer (SP) mutat a "zsák" tetejére (mindig az utoljára betöltött adat).
Előnye, hogy nem kell címezni, így egyszerű, rövid utasításokkal kezelhető és gyors.
Hátránya viszont, hogy mindig csak a tetejéhez férünk hozzá, így az operandus kiolvasás csak szekvenciálisan történhet.
A szekvenciális kiolvasás lassú, szűk keresztmetszetet jelent sok adatnál.

\subsection{Adattípusonként különböző regiszterek}
Itt különféle adattípusokhoz különféle regiszterek állank rendelkezésre, elsősorban lebegőpontos adatoknál.
A lebegőpontos regisztereknél külön van bontva a mantissza és a karakterisztika, mivel mindkettőn műveletet kell végezni, így lehetőséget ad a párhuzamosításra a külön tárolás.
Az Intel 1998-ban bevezette még a SIMD adattípust, amit multimédiás alkalmazásokhoz használtak.

\subsection{Többszörös regiszterkészlet}
Ez a legfejlettebb megoldás, egymásba ágyazott eljárások gyorsítására szolgál.
Ehhez szükséges a kontextus fogalmának bevezetése.
A kontextus a regisztertér állapota az állapottérrel együtt.

Az eljárások közötti váltásoknál szükséges a kontextusok közötti váltás, amit a kontextuskapcsoló végez.
Ha a kontextusok közötti váltásnál a kontextust a memóriatérből kell betölteni, nagyon lassú lesz, ha viszont regiszterek között kell csak váltani, akkor gyors.
Ezért vezettek be több regiszterkészletet, a cél minden kontextus számára különálló regiszterkészlet biztosítása.
A programozó ezek közül csak egyet lát, a kontextus váltásokat a processzor kezeli.
A megvalósításhoz szükség van még egy általános regiszterkészletre, ami a regiszterek közti paraméter átadást biztosítja.

\subsubsection{Típusai}
\begin{itemize}
    \item több, egymástól független regiszterkészlet, hátránya, hogy a paraméter átadás csak az operatív táron keresztül történhet, ami lassítja az eljárást
    \item átfedő regiszterkészlet, egymásba ágyazott eljárásokhoz dolgozták ki, lényege, hogy egy regisztert három részre bontottak: ins, locals, outs. Úgy alakították ki, hogy az egyik regiszter kimenő területe ugyanazon a címen helyezkedik el, mint a másik bemenője, így nagyon könnyű az operátor átadás. Hátránya, hogy merev a struktúra és hogy a regiszterkészletek száma korlátozza a regisztertérben tartható kontextusokat.
    \item stack cache: 1982-ben vezették be, a mai processzorokban is hasonló struktúrákat alkalmaznak. A stack és a közvetlen elérésű cache kombinálása. Kezelése a compiler feladata, ami minden eljáráshoz hozzárendel egy regiszterkészletet. A hozzárendelt regiszterkészlet az aktiválási rekord, aminek az első elemére a stack pointer mutat. A regiszterek nem csak a stack pointeren kersztül érhetők el, hanem közvetlenül is, a displacement pointerrel. Bizonyos korlátok között az aktiválási rekord bármilyen hosszú lehet, tehát a kiosztás rugalmas. Előnye, hogy nincsenek üres helyek, így nincsenek fölöslegesen lefoglalt regiszterek és nincs túlcsordulás se.
\end{itemize}