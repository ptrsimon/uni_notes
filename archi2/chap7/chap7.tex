% 9. előadás vége

\chapter{Szál szinten párhuzamos architektúrák}

\section{Bevezetés}
Az Intel 80386-os processzora óta bevezetett újítások a fogyasztás és a lapkaméret növekedésével jártak, viszont ehhez képest a teljesítmény csak kisebb mértékben javult.
Tehát a teljesítmény növekedése nem állt egyenes arányban a komplexitás növekedésével.
Ez elsősorban az egy magos processzorokra igaz.
Következmény, hogy egyre több tranzisztor kell és nő a fogyasztás.
Ennek megoldásához el kellett térni a klasszikus tervezési elvektől, ahol csak utasítás szinten használták ki a párhuzamosságokat.
A fejlődés következő lépcsőjét a szál szintű párhuzamosság kihasználása jelentette.

\section{A szál}
A szál a program legkisebb önállóan végrehajtható része $\rightarrow$ párhuzamosan futtatható.
Míg az utasítás szintű párhuzamosság felderítésére önmagában képes a hardver, a szálak kihasználásához az operációs rendszer támogatására is szükség van.