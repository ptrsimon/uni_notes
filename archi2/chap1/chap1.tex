% Első előadás

\chapter{Bevezetés}

\section{A tárgy célja}
A tárgy célja a párhuzamos számítógép architektúrák ismertetése.

\section{Történeti áttekintés}
Az iparban először a Ford T-modell gyárása során alkalmazták a párhuzamos végrehajtást (futószalag).
A számítástechnikában is hasonlóan működnek a párhuzamos architektúrák.
Fő cél a fejlesztés során a feldolgozás gyorsítása, ezért a nagy számítógépeknél már a 60-as években megjelent a párhuzamos végrehajtás.
A mikroszámítógépekben ugyanez a 80-as években történt meg.

\section{A párhuzamosság csoportosítása}

\subsection{Funkció szerint}
\begin{itemize}
    \item Rendelkezésre álló párhuzamosság (feladatokban rejlő párhuzamosítási lehetőségek).
    \item Kihasználható párhuzamosság (valóban használható párhuzamosság).
\end{itemize}

\subsection{Típus szerint}
\begin{itemize}
    \item Adat párhuzamosság.
    \item Funkcionális párhuzamosság.
\end{itemize}

\subsection{Elhelyezkedés szerint}
\begin{itemize}
    \item Időbeli párhuzamosság (futószalag): Ford T-modell gyártósor analógia szerint az egyik helyen fényeznek, másik helyen beszerelik a motort egyidőben.
    \item Térbeli párhuzamosság (több, azonos típusú végrehajtó egység egyidejű működése): előző analógia szerint két helyen két különböző autót fényeznek egyszerre.
\end{itemize}

\section{Adat párhuzamosság}
Hasznosítása kétféleképpen lehetséges:
\begin{itemize}
    \item Adatpárhuzamos architektúrákkal (adatelemeken párhuzamos, vagy futószalag elvű műveletek végrehajtása).
    \item Funkcionális párhuzamossággá alakítással (adatelemeken történő műveletek ciklus formájában történő megfogalmazása).
\end{itemize}

\section{Funkcionális párhuzamosság}
A funkcionális párhuzamosság a feladat logikájából következik.
Az architektúrák, operációs rendszerek és a fordítóprogramok is igyekeznek kihasználni.

\subsection{Szintjei}
Különböző szinteken értelmezhetjük (alacsonytól magas felé haladva):
\begin{itemize}
    \item Utasítás szintű párhuzamosság (ILP - Instruction Level Parallelism): program utasítások párhuzamos végrehajtása.
    \item Ciklus szintű párhuzamosság: egymást követő iterációk párhuzamos végrehajtása (függőségek akadályozhatják).
    \item Eljárás szintű párhuzamosság: mértéke leginkább a feladat jellegétől függ.
    \item Program szintű párhuzamosság: egymástól független programok párhuzamos végrehajtása.
    \item Felhasználó szintű párhuzamosság: több, egymástól független felhasználó kiszolgálása (pl. szerverek, időosztásos rendszerek).
\end{itemize}

\subsection{Kihasználása}
\subsubsection{Utasítás szintű párhuzamosság esetén}
\begin{itemize}
    \item Utasítás szinten párhuzamos architektúrákkal.
    \item Erre a célra szolgáló compiler segítségével.
\end{itemize}
\subsubsection{Ciklus- és eljárás szintű párhuzamosság esetén}
Szálak (folyamatok) segítségével.
A szál vagy folyamat a tárgykód legkisebb önállóan végrehajtható része.
Szálakat létrehozhat:
\begin{itemize}
    \item Programozó, párhuzamos nyelveket használva (fork, join...).
    \item Operációs rendszer, ami támogatja a többszálas végrehajtást.
    \item Párhuzamos fordítóprogram.
\end{itemize}
