% Első előadás

\chapter{Bevezetés}

\section{A tárgy célja}
A tárgy célja a párhuzamos számítógép architektúrák ismertetése.

\section{Történeti áttekintés}
Az iparban először a Ford T-modell gyárása során alkalmazták a párhuzamos végrehajtást (futószalag).
A számítástechnikában is hasonlóan működnek a párhuzamos architektúrák.
Fő cél a fejlesztés során a feldolgozás gyorsítása, ezért a nagy számítógépeknél már a 60-as években megjelent a párhuzamos végrehajtás.
A mikroszámítógépekben ugyanez a 80-as években történt meg.

\section{A párhuzamosság csoportosítása}

\subsection{Funkció szerint}
\begin{itemize}
    \item Rendelkezésre álló párhuzamosság (feladatokban rejlő párhuzamosítási lehetőségek).
    \item Kihasználható párhuzamosság (valóban használható párhuzamosság).
\end{itemize}

\subsection{Típus szerint}
\begin{itemize}
    \item Adat párhuzamosság.
    \item Funkcionális párhuzamosság.
\end{itemize}

\subsection{Elhelyezkedés szerint}
\begin{itemize}
    \item Időbeli párhuzamosság (futószalag): Ford T-modell gyártósor analógia szerint az egyik helyen fényeznek, másik helyen beszerelik a motort egyidőben.
    \item Térbeli párhuzamosság (több, azonos típusú végrehajtó egység egyidejű működése): előző analógia szerint két helyen két különböző autót fényeznek egyszerre.
\end{itemize}

\section{Adat párhuzamosság}
Hasznosítása kétféleképpen lehetséges:
\begin{itemize}
    \item Adatpárhuzamos architektúrákkal (adatelemeken párhuzamos, vagy futószalag elvű műveletek végrehajtása).
    \item Funkcionális párhuzamossággá alakítással (adatelemeken történő műveletek ciklus formájában történő megfogalmazása).
\end{itemize}

\section{Funkcionális párhuzamosság}
A funkcionális párhuzamosság a feladat logikájából következik.
Az architektúrák, operációs rendszerek és a fordítóprogramok is igyekeznek kihasználni.

\subsection{Szintjei}
Különböző szinteken értelmezhetjük (alacsonytól magas felé haladva):
\begin{itemize}
    \item Utasítás szintű párhuzamosság (ILP - Instruction Level Parallelism): program utasítások párhuzamos végrehajtása.
    \item Ciklus szintű párhuzamosság: egymást követő iterációk párhuzamos végrehajtása (függőségek akadályozhatják).
    \item Eljárás szintű párhuzamosság: mértéke leginkább a feladat jellegétől függ.
    \item Program szintű párhuzamosság: egymástól független programok párhuzamos végrehajtása.
    \item Felhasználó szintű párhuzamosság: több, egymástól független felhasználó kiszolgálása (pl. szerverek, időosztásos rendszerek).
\end{itemize}

\subsection{Kihasználása}
\subsubsection{Utasítás szintű párhuzamosság esetén}
\begin{itemize}
    \item Utasítás szinten párhuzamos architektúrákkal.
    \item Erre a célra szolgáló compiler segítségével.
\end{itemize}
\subsubsection{Ciklus- és eljárás szintű párhuzamosság esetén}
Szálak (folyamatok) segítségével.
A szál vagy folyamat a tárgykód legkisebb önállóan végrehajtható része.
Szálakat létrehozhat:
\begin{itemize}
    \item Programozó, párhuzamos nyelveket használva (fork, join...).
    \item Operációs rendszer, ami támogatja a többszálas végrehajtást.
    \item Párhuzamos fordítóprogram.
\end{itemize}
\subsubsection{Szemcsézettségi szintek}
A funkcionális párhuzamosság kihasználásához különböző szemcsézettségi szintek rendelhetők.
A finom szemcsézettségtől a durváig haladva a párhuzamossági szintek:
\begin{itemize}
    \item utasításszint
    \item szál szint
    \item folyamat szint
    \item felhasználói szint
\end{itemize}
A finom szemcsézettségű párhuzamosság alacsony szinten, közvetlenül kihasználható.
A magasabb szintű (durvább szemcsézettségű) párhuzamosság kihasználásához nem elég a hardver és a compiler támogatása, hanem OS-ek alatti konkurens végrehajtás szükséges.

\section{Compilerek}
A compilerek a magas szintű nyelveken írt programokat fordítja le a processzor által értelmezhető kódra.

\subsection{Feladatai}
\begin{itemize}
    \item analizálni a forrásnyelvű program karaktersorozatát,
    \item szintetizálni (létrehozni) a tárgykódot.
\end{itemize}

\subsection{Analízis részei}
\begin{itemize}
    \item Lexikális elemzés (konstansok, változók, operátorok).
    \item Szintaktikai elemzés.
    \item Szemantikai elemzés.
\end{itemize}
\subsection{Szintézis részei}
\begin{itemize}
    \item Kódgenerálás (assembly kód vagy gépi kód).
    \item Kód optimalizálás (pl. független részek keresése párhuzamos végrehajtáshoz).
\end{itemize}

\section{Párhuzamos architektúrák osztályozása}
\subsection{Klasszikus modell}
A klasszikus osztályozási modell J. Flynn amerikai mérnök nevéhez kötődik.
Ez a rendszer a vezérlő és a feldolgozási egységek számán alapul.
Az osztályozáshoz bevezetett új fogalmak:
\begin{itemize}
    \item SI (Single Instruction stream) - egyszeres utasításfolyam: az architektúra egy vezérlőegységgel rendelkezik.
    \item MI (Multiple Instruction stream) - többszörös utasításfolyam: az architektúra több utasításfolyamot tud egyidőben végrehajtani.
    \item SD (Single Data stream) - egyszeres adatfolyam: egy CPU egy adatfolyamot dolgoz fel.
    \item MD (Multiple Data stream) - többszörös adatfolyam: több végrehajtó egység egy időben dolgoz fel több, egymástól független adatfolyamot.
\end{itemize}
Ezek alapján a következő architektúra osztályok határozhatók meg:
\begin{itemize}
    \item SISD (Single Instruction Single Data): Neumann modell, szekvenciális végrehajtás.
    \item SIMD (Single Instruction Multiple Data): multimédiás feldolgozás, ugyanazon műveletek végrehajtása sok adaton.
    \item MISD (Multiple Instruction Single Data): elméleti, nem használják a gyakorlatban.
    \item MIMD (Multiple Instruction Multiple Data): teljes párhuzamos feldolgozás.
\end{itemize}
A Flynn modell egyszerű és átlátható, de nem utal a párhuzamosság fajtájára, szintjére és módjára.

\subsection{Új modell}
Az új osztályozás megkülönböztet adatpárhuzamos és funkcionálisan párhuzamos architektúrákat.
\subsubsection{Adatpárhuzamos architektúrák}
Az adatpárhuzamos architektúrák típusai:
\begin{itemize}
    \item verktor
    \item asszociatív (neurális)
    \item SIMD
    \item szisztorikus
\end{itemize}
A gyakorlatban leggyakrabban a SIMD architektúrákat használják, a tárgy is csak erre tér ki.
\subsubsection{Funkcionálisan párhuzamos architektúrák}
A funkcionálisan párhuzamos architektúrák típusai:
\begin{itemize}
    \item utasítás szinten párhuzamos (ILP)
    \begin{itemize}
        \item futószalag (időbeli párhuzamosság)
        \item VLIW (térbeli+időbeli párhuzamosság)
        \item szuperskalár (térbeli+időbeli párhuzamosság)
    \end{itemize}
    \item szál szinten párhuzamos (SMT - Simultaneous multithreading), pl. Intel Hyper Threading
    \item folyamat szinten párhuzamos
    \begin{itemize}
        \item elosztott memória használatú
        \item közös memória használatú
    \end{itemize}
\end{itemize}

\section{Történeti áttekintés}
A desktop gépekben először a 80-as években jelent meg a párhuzamosság, futószalag elvű processzorokkal.
Ezek voltak az első ILP processzorok, skalár CPU-nak is hívták őket.
A szuperskalár processzorok a 90-es évek elején jelentek meg, majd később kiegészültek multimédiás kiterjesztéssel.
A szuperskalár processzorok dinamikus utasítás ütemezéssel rendelkeztek, így elérték a sorrenden kívüli kibocsátást és a kibocsátási párhuzamosságot.
A 2000-es évektől a fejlődésnek két iránya jelent meg: evolúciós és revolúciós.
\paragraph{Evolúciós fejlődés:} a logikai architektúra változatlan, viszont a feldolgozási szélesség 32-ről 64 bitre nőtt.
Ennek eredménye az x86-os architektúra bővülése x86\_64-re.
\paragraph{Revolúciós fejlődés:} ez egy teljesen új logikai architektúrát és utasításkészletet eredményezett.
Ez az Intel és a HP által kifejlesztett VLIW elvű, IA64 architektúra.
\paragraph{Eredmény:} a fejlődés eredményeként megjelentek a többmagos processzorok, a mai napig ez az irány a meghatározó.

\section{Utasítások felbontása}
A processzor által végrehajtott utasítások négy elemi műveletre oszthatók:
\begin{itemize}
    \item Fetch
    \item Decode
    \item Execute
    \item WriteBack
\end{itemize}

\section{Kibocsátás és végrehajtás}
A kibocsátás a processzor dekódoló egységéből történik, ezeket a kibocsátott utasításokat hajtják végre a végrehajtó egységek.
Ha a CPU képes párhuzamosan több utasítás végrehajtására, a kibocsátási kapacitást is növelni kell.
Ha a CPU a dekódoló egységéből óraciklusonként 1-nél több utasítás kibocsátására képes, megvalósul a kibocsátási párhuzamosság.
\paragraph{Következmények:}
\begin{itemize}
    \item Az utasítások párhuzamos végrehajtása során minden ILP CPU-nak figyelembe kell vennie az utasítások között fennálló függőségeket.
    \item Meg kell őrizni a soros végrehajtás konzisztenciáját (a programozó soros végrehajtású programot ír, ennek párhuzamos végrehajtás esetén is ugyanúgy kell viselkednie).
\end{itemize}