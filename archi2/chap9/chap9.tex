% 11. előadás vége, 12. előadás eleje

\chapter{A DDR4 memória}

\section{Bevezetés}
A DDR3 memóriák viszonylag alacsony sávszélessége és magas fogyasztása miatt szükségessé vált a RAM-ok továbbfejlesztése, így született meg a DDR4.

\section{Összehasonlítás}
A DDR4-es memóriák alacsonyabb feszültségen üzemelnek, így kevesebbet fogyasztanak.
Ezen kívül több pines csatlakozót használ, így nem kompatibilis a DDR3-al.
Mindkettő 8n prefetch eljárást alkalmaz.
Újdonság a DDR4-nél a bank groupok létrehozása.
Míg a DDR3-nál 1 chip 8 különálló bankot tartalmaz, DDR4-nél 1 chip 4 db 4 bankból álló csoportot.
\begin{center}
    \begin{tabular}{c | c | c}
            & DDR3 & DDR4 \\
        \hline
        Feszültség & 1,5-1,35V & 1,2V \\
        \hline
        Csatlakozó & 240 pin & 284 pin \\
        \hline
        Prefetch & 8n & 8n \\
        \hline
        Bankok & 8 x 1 & 4 x 4 \\
    \end{tabular}
\end{center}

\section{A bank groupok}
A bankok csoportosításának előnye, hogy a groupokból a vezérlés egyszerre kettőt vagy négyet is kiválaszthat, ami a sávszélesség növekedéséhez vezet (mivel a memória műveletek párhuzamosan futnak le).

\section{Prefetch}
A prefetch technológia két okból maradt ugyanúgy 8n:
\begin{itemize}
    \item a 16n-hez még több pinre lett volna szükség és
    \item a 16n nem illeszkedik a 64 byte-os cache vonal mérethez (ez csökkentette volna a teljesítményt).
\end{itemize}

\section{További újítások}
\begin{itemize}
    \item CRC (Cyclic Redundancy Check) ellenőrzés $\rightarrow$ véletlenszerű anomáliákat is érzékeli (és akár javítja is), biztosabb az adat olvasása.
    \item Chipenkénti extra paritás.
    \item ODT (On-Die Termination): chipenkénti lezárás, a feszültség szabályozásban segít, így alacsonyabb feszültségen is stabilan működik a memória.
\end{itemize}

\section{Értékelés}
A kisebb feszültség eléréséhez szükséges volt a memóriacellák órajelének csökkentése a DDR3-hoz képest, ez látható a következő táblázatban:
\begin{center}
    \begin{tabular}{c | c | c | c | c}
        Effektív órajel & f\textsubscript{core} & Max. sávszélesség & Késleltetés (időzítés) & Elérési idő \\
        \hline
        DDR3-2133 MHz & 266 MHz & 17066 MB/s & 14-14-14-34 & \~13,1 ns \\
        \hline
        DDR4-2133 MHz & 133 MHz & 17066 MB/s & 15-15-15-35 & \~14,1 ns \\
        \hline
        DDR4-2400 MHz & 150 MHZ & 19200 MB/s & 16-16-16-39 & \~13,3 ns \\
    \end{tabular}
\end{center}
Látható, hogy a DDR4-es memóriák alacsonyabb frekvencia mellett nagyobb sávszélességet tudnak biztosítani, mint a DDR3, ugyanakkor növekedett a késleltetés és az elérési idő.
A cache-ek használata (L3) miatt viszont a sávszélesség fontosabb, mint a késleltetés mértéke, ezért a gyakorlati alkalmazásokban (pl. multimédia) a DDR4 előrelépést jelentett.
További előnye a DDR4-nek, hogy az effektív órajel tovább növelhető az alacsonyabb magfrekvencia miatt.
Míg a DDR3 felső korlátja 2133 MHz, a DDR4 3000 MHz fölé is mehet.